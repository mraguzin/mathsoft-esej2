\documentclass[12pt]{scrartcl}
\usepackage[utf8]{inputenc}
\usepackage[croatian]{babel}
\usepackage[unicode]{hyperref}
\usepackage{amsmath,amssymb,amsthm}
\usepackage{mathtools}
\usepackage{thmtools}
\usepackage{csquotes}
\usepackage[backend=biber]{biblatex}
\addbibresource{literatura.bib}
\MakeOuterQuote{"}
\declaretheorem{teorem}
\declaretheorem[style=definition,sibling=teorem,qed=$\vartriangleleft$]{definicija}

\newcommand{\T}{^\mathsf T}
\newcommand{\mat}[1]{%
    \ifmmode%
        \mathbf{#1}%
    \else%
        $\mathbf{#1}$%
    \fi%
}
\newcommand{\vek}[1]{\mat{#1}}

\author{Mauro Raguzin}
\title{Korijen --- 1 algoritam, 3 implementacije}
\date{\today}

\begin{document}
\maketitle
\tableofcontents
\pagebreak

\section{Uvod}
%V1:Ovaj esej se bavi naoko jednostavnim i elementarnim numeričkim algoritmom: računom kvadratnog korijena. Iako u osnovi poznat tisućama godina,
%ovaj algoritam ima fascinantan broj varijacija, što u njegovoj osnovnoj konceptualizaciji neovisnoj o namijenjenom izvršitelju, što u konkretnim
%formama koje dobiva u ovisnosti o različitim računalnim arhitekturama na kojima se povijesno implementirao.
Izračunati kvadratnoi korijen --- koliko bi to moglo biti teško? Iako u osnovi poznat tisućama godina,
ovaj algoritam ima fascinantan broj varijacija, što u njegovoj osnovnoj konceptualizaciji neovisnoj o namijenjenom izvršitelju, što u konkretnim
formama koje dobiva u ovisnosti o različitim računalnim arhitekturama na kojima se povijesno implementirao.

U ovom eseju obrađujemo u detalje ovaj algoritam kroz povijest, od TODO!! do današnjeg kompjutoriziranog doba u kojem su precizne i vrlo efikasne varijante
ovog algoritma od presudne važnosti u najrazličitijim poljima primjene, poput simulacije, digitalne fizike i računalne grafike. Usput ćemo predstaviti
i neke analize posebno zanimljivih modernih računalnih implementacija te predstaviti mjerenja obavljena na današnjim računalima za različite implementacije.
% TODO: zašto baš tri implementacije? Navedi tu ukratko razlog + koliko smo ih ustvari točno obradili!

\section{Povijest računa kvadratnog korijena}
\begin{quotation}
    Seeing there is nothing that is so troublesome to Mathematicall practise, nor that doth more molest and hinder Calculators,
    then the Multiplications, Diuisions, square and cubical Extractions of great numbers, which besided the tedious expence
    of time, are for the most part subiect to many slippery errors. I began therefore to consider in my minde, by what certaine
    and ready Art I might remoue those hindrances.\cite[str.~194]{taocp2}
    \begin{flushright}
        ---John Napier (1616.)
    \end{flushright}
\end{quotation}


\printbibliography

\end{document}